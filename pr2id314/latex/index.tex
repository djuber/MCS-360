\hyperlink{classPolynomial}{Polynomial} Manipulation Program. MCS 360 Fall 2011 Project 2\hypertarget{index_Overview}{}\section{Overview}\label{index_Overview}
This is a menu driven program which allows a user to input polynomials of arbitrary length and degree. Operations including multiplication and addition of polynomials is available for any previously defined polynomials, and all results are saved for future use.\hypertarget{index_Use}{}\section{Use}\label{index_Use}
To begin, add polynomials to work with. The input routine first asks for the number of terms in the polynomial, and then for each term requests a degree and a coefficient. The resulting polynomial is the sum of the given terms. Polynomials are always stored and displayed in ascending degree, and reduced by combining like terms.

Once polynomials have been added to the system, they may be recalled by numbers (in order of creation). Unneeded terms may be deleted when the number of polynomials in memory makes using the system unpleasant.

Polynomials may be added, subtracted, or multiplied. A special case for multiplying by a scalar is allowed, which is simply a constant polynomial under the hood.\hypertarget{index_Cavaets}{}\section{Cavaets}\label{index_Cavaets}
It should be noted that multiplying two very large polynomials (number of terms) is a time consuming process O(n$^\wedge$2), while addition and evaluation at a point are O(n).

Scalar multiplication cowardly refuses to multiply by zero. The user is welcome to solve this himself.

Coefficients are represented using machine doubles. Repeated multiplication or addition will result in {\em inf\/} as a value. Precision is inversely proportional to magnitude for doubles. No attempt to minimize errors was made.\hypertarget{index_History}{}\section{History}\label{index_History}
This was the second programming project for MCS 360 at UIC in Fall 2011. Substantial portions of the menu system and the user input routines were adapted from the first project. Any future revisions would do well to modularize further the menu object, and possibly allow reassignment of the istream/ostream for the \hyperlink{classInputValidator}{InputValidator}. 